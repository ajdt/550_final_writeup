\documentclass{beamer}
\usetheme{default}
\usepackage{shortcuts}

\title[Abstract Pattern Formation]{A Randomized Approach to Abstract Pattern Formation Fat Robots}
%\subtitle[Errors]{Estimation of numerical errors}
\author[A. Diaz Tolentino \& S. Ramamoorthy]{Armando Diaz Tolentino \& Siva Ramamoorthy}
\institute[UW]{
	Department of Computer Science\\
	University of Washington\\
	Seattle, Washington \\[1ex]
	\texttt{\{ajdt, sivanr\}@cs.washington.edu}
}
\begin{document}

\begin{frame}
	\titlepage	
\end{frame}
\begin{frame}
	\tableofcontents
\end{frame}

\begin{frame}{Problem Introduction}
	\textbf{Arbitrary Pattern Formation (APF):} 
	\begin{itemize}
		\item Given: A set of $n$ robots arranged arbitrarily on 2D plane 
		\item Goal: form an arbitrary pattern known by all robots \textit{a priori}
		\item Operations: Look, Move, Wait
		\item No explicit communication between robots
	\end{itemize}

	Various Modelling Choices:
	\begin{itemize}
		\item history oblivious, synchronous (FSYNC), ansynchronous (ASYNC)
		\item point vs disk model, transparent vs opaque
	\end{itemize}

	Assumptions:
	\begin{itemize}
		\item compass orientations: 
		<insert graphic here>
	\end{itemize}

\end{frame}
\begin{frame}{Our Model}
	\begin{itemize}
		\item unit disk, opaque, history oblivious, ASYNC
	\end{itemize}


\end{frame}

\begin{frame}{Notation}
	\begin{itemize}
		\item We refer to a configuration, $\EE$, as an $n$-tuple of points: $\EE = (x_1,\ldots x_n)$, $\x_i \in \RR^2$
		\item 
	\end{itemize}
\end{frame}

\begin{frame}{APF and Leader Election}
	insert slide content	
\end{frame}

\begin{frame}{Prior Work}
	insert slide content	
\end{frame}

\begin{frame}{Our Results: Randomized APF}
	insert slide content
\end{frame}

\begin{frame}{PUDDLE: }
	insert slide content
\end{frame}

\begin{frame}{A sample slide}

%A displayed formula:

%\[
%  \int_{-\infty}^\infty e^{-x^2} \, dx = \sqrt{\pi}
%\]

An itemized list:

\begin{itemize}
  \item itemized item 1
  \item itemized item 2
  \item itemized item 3
\end{itemize}

\begin{theorem}
  In a right triangle, the square of hypotenuse equals
  the sum of squares of two other sides.
\end{theorem}

\end{frame}

\end{document}
