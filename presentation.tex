\documentclass{beamer}
\usetheme{default}
\usepackage{shortcuts}

\title[Abstract Pattern Formation]{A Randomized Approach to Abstract Pattern Formation Fat Robots}
%\subtitle[Errors]{Estimation of numerical errors}
\author[A. Diaz Tolentino \& S. Ramamoorthy]{Armando Diaz Tolentino \& Siva Ramamoorthy}
\institute[UW]{
	Department of Computer Science\\
	University of Washington\\
	Seattle, Washington \\[1ex]
	\texttt{\{ajdt, sivanr\}@cs.washington.edu}
}
\begin{document}

\begin{frame}
	\titlepage	
\end{frame}
\begin{frame}
	\tableofcontents
\end{frame}

\begin{frame}{Problem Introduction}
	\textbf{Arbitrary Pattern Formation (APF):} 
	\begin{itemize}
		\item Given: A set of $n$ robots arranged arbitrarily on 2D plane 
		\item Goal: form an arbitrary pattern known by all robots \textit{a priori}
		\item Operations: Look, Move, Wait
		\item No explicit communication between robots
	\end{itemize}

	Related Problems:
	\begin{itemize}
	% todo: insert graphics to make these problems clear
		\item gathering
		\item pattern sequencing
		\item flocking
	\end{itemize}
\end{frame}

% insert slide for motivations

\begin{frame}{Modelling Choices}
	Various Modelling Choices:
	\begin{itemize}
		\item history oblivious, synchronous (FSYNC), ansynchronous (ASYNC)
		\item point vs disk model, transparent vs opaque
	\end{itemize}

	Assumptions:
	\begin{itemize}
		\item compass orientations: 
		\begin{itemize}
			\item consistent compass
			\item OneAxis
			\item Chirality
		\end{itemize} 
		\item rescaling and rotation of shape
		\item number of robots
		<insert graphics here to explain concepts>
	\end{itemize}
\end{frame}

\begin{frame}{Notation}
	\begin{itemize}
		\item We refer to a configuration, $\EE$, as an $n$-tuple of points: $\EE = (x_1,\ldots x_n)$, $x_i \in \RR^2$
		\item $\EE_i$ is configuration seen by robot $i$
		\item $\PP$ target pattern robots must form
	\end{itemize}
\end{frame}

\begin{frame}{Prior Work: Impossibility Results}
	\begin{theorem} 
		APF is not solvable deterministically for an even number of robots $ n > 2$.
	\end{theorem} 
	INSERT GRAPHIC SHOWING WHY
	\begin{theorem} 
		APF is solvable deterministically for odd number of robots even in ASYNC.
	\end{theorem} 
	INSERT GRAPHIC SHOWING WHY
\end{frame}

\begin{frame}{Prior Work: APF and Leader Election}
	\begin{theorem} 
		APF is solvable for $n \geq 3$ robots $\iff$ Leader election problem is solvable.
	\end{theorem} 
	APF $\rightarrow$ leader election:
	\pause
	INSERT GRAPHIC HERE

	leader election $\rightarrow$ APF: 
	\pause
	INSERT GRAPHIC HERE

	Using this insight, our algorithms work to solve leader election first.

\end{frame}

\begin{frame}{Our Results: Randomized APF}
	Our Model:
	\begin{itemize}
		\item unit disk, opaque, history oblivious, ASYNC
	\end{itemize}
\end{frame}

\begin{frame}{PUDDLE: }
	insert slide content
\end{frame}

\end{document}
